


\begin{overpic}[width=1\textwidth]{Untitled.eps}
 \put (5,90) {\Large{Data analysis:}}
 \put (5,82) {\centering \Huge{Core skills for reproducible }}
 \put (5,76) {\centering \Huge{data analysis in R}}
\end{overpic}

\newpage
{\color{white}text}

\newpage


\section*{Blurb}

This short course covers some of the core skills required for a budding R user to develop a strong foundation for data analysis in the RStudio environment. Within the framework of a reproducible research workflow we will cover importing and cleaning data, efficient coding practices, writing your own functions and using the powerful \texttt{dplyr} data manipulation tools. 

\section*{Key Topics}

\begin{itemize}
\item Reproducible Research
\item R Studio and project management
\item Importing and cleaning data
\item Good coding practices in R
\item Standard control structures
\item Vectorisation and \texttt{apply} functions
\item Writing your own functions
\item Data manipulation with \texttt{dplyr}
\item Piping/chaining commands
\end{itemize}

\section*{Course information}

\emph{Intended audience:}	Anyone interested in quantitative data analysis using open source tools.

\emph{Prior knowledge;} Knowledge of R (as covered in R: An introduction).

\emph{Resources:}	Course handbook and GitHub repository available at http://tinyurl.com/RCSRepRes

\emph{Software:} RStudio \& 	R 3.1.2

\emph{Format;}	Presentation with practical exercises

\emph{Where next?}	Data visualisation: Creating interactive visualisations using R and Shiny course

\subsection*{Document Information}

\emph{Date:} March 2017

\emph{Version:} 1.1a

\emph{Licence:} Maja Zalo\v znik makes this document and any accompanying slides available under a Creative Commons licence – Attribution-NonCommercial- ShareAlike (CC-BY-NC-SA).


\newpage
{\color{white}text}

\newpage